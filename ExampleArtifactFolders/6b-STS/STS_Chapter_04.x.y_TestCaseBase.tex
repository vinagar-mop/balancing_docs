%%% SVN stuff
\svnid{$Id: STS_Chapter_04.x.y_TestCaseBase.tex 128 2024-07-14 20:00:16Z KneadProject $}

\subsection{\StsTestCaseID}
\label{loc:Test\StsTestCaseID}
%%% SVN stuff
\svnid{$Id: STS_4.x.y.0_DIDINFO.tex 126 2024-07-14 03:51:40Z KneadProject $}

\DIDINFO{STS-4.x.y.0 :: This section shall identify a test case by project-unique identifier, state its purpose, and provide a brief description of the test case as it relates to the overall test.
The following subsections shall provide a detailed description of the test case.}

Test \StsTestSpecID{} case \StsTestCaseID{} is \TBD.

\subsubsection{Requirements Addressed}
\label{loc:TestCaseRequirementsAddressed\StsTestSpecID\StsTestCaseID}
%%% SVN stuff
\svnid{$Id: STS_4.x.y.1_DIDINFO.tex 126 2024-07-14 03:51:40Z KneadProject $}

\DIDINFO{STS-4.x.y.1 :: This paragraph shall identify the configuration items or system requirements addressed by the test case.
Alternatively, this information may be provided in the test case description and/or Chapter 5.}

This requirements validated by test \StsTestSpecID{} case \StsTestCaseID{} are listed in \S~\ref{loc:TestCaseProcedure\StsTestSpecID\StsTestCaseID}.

\subsubsection{Prerequisite Conditions}
\label{loc:TestCasePrerequisiteConditions\StsTestSpecID\StsTestCaseID}
%%% SVN stuff
\svnid{$Id: STS_4.x.y.2_DIDINFO.tex 126 2024-07-14 03:51:40Z KneadProject $}

\DIDINFO{STS-4.x.y.2 :: This paragraph shall identify any prerequisite conditions that must be established prior to performing the test case. 
The following considerations shall be discussed, as applicable:
\begin{enumerate}[itemindent=5pt,topsep=0pt,itemsep=0pt,partopsep=0pt, parsep=0pt, label=(\alph*)]
\item hardware, firmware, and/or software configuration, 
\item Flags, initial breakpoints, pointers, control parameters, or initial data to be set/reset prior to test commencement,
\item Preset hardware conditions or electrical states necessary to run the test case,
\item Initial conditions to be used in making timing measurements,
\item Conditioning of the simulated environment, and
\item Other special conditions peculiar to the test case.
\end{enumerate}}

This prerequisite conditions for test \StsTestSpecID{} case \StsTestCaseID{} are \TBD.

\subsubsection{Inputs}
\label{loc:TestCaseInputs\StsTestSpecID\StsTestCaseID}
%%% SVN stuff
\svnid{$Id: STS_4.x.y.3_DIDINFO.tex 126 2024-07-14 03:51:40Z KneadProject $}

\DIDINFO{STS-4.x.y.3 :: This paragraph shall describe the test inputs necessary for the test case.
The following shall be provided, as applicable:
\begin{enumerate}[itemindent=5pt,topsep=0pt,itemsep=0pt,partopsep=0pt, parsep=0pt, label=(\alph*)]
\item Name, purpose, and description (e.g., range of values, accuracy) of each test input,
\item Source of the test input and the method to be used for selecting the test input,
\item Whether the test input is real or simulated,
\item Time or event sequence of test input, and
\item The manner in which the input data will be controlled (See DID for more information).
\end{enumerate}}

This inputs for test \StsTestSpecID{} case \StsTestCaseID{} are listed in \S~\ref{loc:TestCaseProcedure\StsTestSpecID\StsTestCaseID}.

\subsubsection{Expected Outputs}
\label{loc:TestCaseExpectedOutputs\StsTestSpecID\StsTestCaseID}
%%% SVN stuff
\svnid{$Id: STS_4.x.y.4_DIDINFO.tex 126 2024-07-14 03:51:40Z KneadProject $}

\DIDINFO{STS-4.x.y.4 :: This paragraph shall identify all expected test results for the test case. 
Both intermediate and final test results shall be provided, as applicable.}

This expected outputs for test \StsTestSpecID{} case \StsTestCaseID{} are listed in \S~\ref{loc:TestCaseProcedure\StsTestSpecID\StsTestCaseID}.

\subsubsection{Evaluation Criteria}
\label{loc:TestCaseEvaluationCriteria\StsTestSpecID\StsTestCaseID}
%%% SVN stuff
\svnid{$Id: STS_4.x.y.5_DIDINFO.tex 126 2024-07-14 03:51:40Z KneadProject $}

\DIDINFO{STS-4.x.y.5 :: This paragraph shall identify the criteria to be used for evaluating the intermediate and final results of the test case. 
For each test result, the following information shall be provided, as applicable:
\begin{enumerate}[itemindent=5pt,topsep=0pt,itemsep=0pt,partopsep=0pt, parsep=0pt, label=(\alph*)]
\item The range or accuracy over which an output can vary and still be acceptable,
\item Minimum number of combinations or alternatives of input and output conditions that constitute an acceptable test result,
\item Maximum/minimum allowable test duration, in terms of time or number of events,
\item Maximum number of interrupts, halts, or other system breaks that may occur,
\item Allowable severity of processing errors,
\item Conditions under which the result is inconclusive and re-testing is to be performed,
\item Conditions under which the outputs are to be interpreted as indicating irregularities in input test data, in the test database/data files, or in test procedures,
\item Allowable indications of the control, status, and results of the test and the readiness for the next test case (may be output of auxiliary test software), and
\item Additional criteria not mentioned above.
\end{enumerate}}

This evaluation criteria for test \StsTestSpecID{} case \StsTestCaseID{} are listed in \S~\ref{loc:TestCaseProcedure\StsTestSpecID\StsTestCaseID}.

\subsubsection{Assumptions and Constraints}
\label{loc:TestCaseAssumptions\StsTestSpecID\StsTestCaseID}
%%% SVN stuff
\svnid{$Id: STS_4.x.y.6_DIDINFO.tex 126 2024-07-14 03:51:40Z KneadProject $}

\DIDINFO{STS-4.x.y.6 :: This paragraph shall identify any assumptions made and constraints or limitations imposed in the description of the test case due to system or test conditions, such as limitations on timing, interfaces, equipment, personnel, and database/data
files. 
If waivers or exceptions to specified limits and parameters are approved, they shall be identified and this paragraph shall address their effects and impacts upon the test case.}

This procedure for test \StsTestSpecID{} case \StsTestCaseID{} is \TBD.

\subsubsection{Procedure}
\label{loc:TestCaseProcedure\StsTestSpecID\StsTestCaseID}
%%% SVN stuff
\svnid{$Id: STS_4.x.y.7_DIDINFO.tex 126 2024-07-14 03:51:40Z KneadProject $}

\DIDINFO{STS-4.x.y.7 :: This paragraph shall define the test procedure for the test case. 
The test procedure shall be defined as a series of individually numbered steps listed sequentially in the order in which the steps are to be performed. 
For convenience in document maintenance, the test procedures may be included as an appendix and referenced in this paragraph. 
The appropriate level of detail in each test procedure depends on the type of system or subsystem being tested.}

This procedure for test \StsTestSpecID{} case \StsTestCaseID{} is \TBD.

See step~\ref{loc:Step2} for how to reference specific steps.

\TestProcedure%[6] arguments, denoted %N%-NAME
%%%%%%%%%%%%%%%%%%%%%%%%%%%%%%%%%%%%%%%%%
%arg-1 is test procedure number, normally made from section#.X
{%1%-PROCNUM
\getcurrentref{subsubsection}.1
}%1%-PROCNUM
%%%%%%%%%%%%%%%%%%%%%%%%%%%%%%%%%%%%%%%%%
%arg-2 is test procedure name
{%2%-PROCNUM
Test Procedure 1
}%2%-PROCNUM
%%%%%%%%%%%%%%%%%%%%%%%%%%%%%%%%%%%%%%%%%
%arg-3 is test procedure label, for use in references to the test procedure
{loc:TestProc1}
%
%arg-4 is list of requirements validated in this test procedure label
{%4%-RQMTS
\tpRqmt{Fake Rqmt 1}
\tpRqmt{Fake Rqmt 2}
}%4%-RQMTS
%%%%%%%%%%%%%%%%%%%%%%%%%%%%%%%%%%%%%%%%%
%arg-5 is list of notes for this test procedure label
{%5%-NOTES
\tpNote{Note 1}
\tpNote{Note 2}
}%5%-NOTES
%%%%%%%%%%%%%%%%%%%%%%%%%%%%%%%%%%%%%%%%%
%arg-6 is list of steps for this test procedure label
% provided as a list of special commands tpStepXYZ[2], tpStep[3], or tpStepLabeled[4]
{%6%-STEPS
%
%RECORD PRE-TEST INFORMATION
%%% SVN stuff
\svnid{$Id: KNEAD_PreTestInfo.tex 126 2024-07-14 03:51:40Z KneadProject $}

\tpStepINFO{START RECORDING OF PRE-TEST INFORMATION}
%
\tpStep%{Action}{Expected Result}{Space to record results}
{Record Date and Time at Start of Test}
{Date and Time at Start of Test are recorded.}
{40pt}
%
\tpStep%{Action}{Expected Result}{Space to record results}
{Record Name of Test Engineer(s) and Agency}
{Name of Test Engineer(s) and their Agency are recorded.}
{40pt}
%
\tpStep%{Action}{Expected Result}{Space to record results}
{Record Name of Witness(es) and Agency}
{Name of Witness(es) and their Agency are recorded.}
{40pt}
%
\tpStep%{Action}{Expected Result}{Space to record results}
{Record configuration information or name of file that contains such information.}
{Configuration information or name of file that contains such information is recorded.}
{100pt}
%
\tpStepBULLSEYE{END RECORDING OF PRE-TEST INFORMATION}
%from $TEXINPUTS; make a local copy and adjust as appropriate.
%
% use these as section delineators as needed; pick image that best fits the need
\tpStepCLOCK{CLOCK TEXT -- good to show that some time must elapse} 
\tpStepHAND{HAND TEXT -- use, with text, when tester needs to pause and double check things} 
\tpStepINFO{INFO TEXT -- good to provide information needed at this point in the test}
\tpStepKEY{KEY TEXT -- good to make a key point, or if something needs to be locked/unlocked}
\tpStepMAGNIFY{MAGNIFY TEXT -- note info that magnifies what is happening} 
\tpStepPLAYARROW{PLAYARROW TEXT -- denote a starting point, such as when test stations change.
This is just more text to see what happens when there are 3 or 4 lines of text w.r.t. centering of icon.
These text blocks should be short, but, could be long, so this checks to see what happens with 5 or 6 lines of text.}
\tpStepBANG{BANG TEXT -- denote a WARNING}% 
\tpStepSHOCK{SHOCK TEXT -- denote a HAZARD}
\tpStepRADIATION{RADIATION TEXT -- denote an EXTREME HAZARD} 
\tpStepONOFF{ON/OFF TEXT -- denote a POWER ON or OFF action} 
\tpStepFOOTSTEPS{FOOTSTEPS TEXT -- denote a MOVEMENT action} 
\tpStepREPEAT{REPEAT TEXT -- denote a repeat cycle or action} 
\tpStepQUESTION{QUESTION TEXT -- ensure a question is answered} 
\tpStepBULLSEYE{BULLSEYE TEXT -- denote the end of a mini-sequence}
\tpStepFILES{FILES TEXT -- denote storage of data or records}
%
\tpStep%{Action}{Expected Result}{Space to record results}
{Do Step 1}
{Step 1 works!}
{20pt}
%
\tpStepLabeled%{Action}{Expected Result}{Space to record results}{label for use in \reference{}}
{Do Labeled Step 2}
{Step 2 works!}
{20pt}
{loc:Step2}
%
% This one shows how to insert an image in the test step;
% most like to show expected results, but can be used as test step to show the step if it makes sense
\tpStepLabeled%{Action}{Expected Result}{Space to record results}{label for use in \reference{}}
{Do stuff so the system looks like the expected image below.}
{\ifpdf
\tpStepFigure{images/KNEAD_UnderConstruction_Image.pdf}{1.0in}
\else
\tpStepFigure{images/KNEAD_UnderConstruction_100dpi_6.5inchesWide.eps}{1.0in}
\fi}{20pt}
{loc:Step3}
%
% This one shows how to insert an image in the test step;
% most like to show expected results, but can be used as test step to show the step if it makes sense
\tpStepLabeled%{Action}{Expected Result}{Space to record results}{label for use in \reference{}}
{\ifpdf
\tpStepFigure{images/KNEAD_UnderConstruction_Image.pdf}{1.0in}
\else
\tpStepFigure{images/KNEAD_UnderConstruction_100dpi_6.5inchesWide.eps}{1.0in}
\fi}
{Labeled step 4 works!}
{20pt}
{loc:Step4}
%
%RECORD POST-TEST INFORMATION
%%% SVN stuff
\svnid{$Id: KNEAD_PostTestInfo.tex 126 2024-07-14 03:51:40Z KneadProject $}

\tpStepINFO{START RECORDING OF POST-TEST INFORMATION}
%
\tpStep%{Action}{Expected Result}{Space to record results}
{Record Date and Time at End of Test}
{Date and Time at End of Test are recorded.}
{40pt}
%
\tpStep%{Action}{Expected Result}{Space to record results}
{Record Signature of Test Engineer(s)}
{Signature of Test Engineer(s) is recorded.}
{40pt}
%
\tpStep%{Action}{Expected Result}{Space to record results}
{Record Signature of Witness(es)}
{Signature of Witness(es) are recorded.}
{40pt}
%
\tpStep%{Action}{Expected Result}{Space to record results}
{Record any pertinent comment about the test procedure, results, and/or environment.}
{Any pertinent comment about the test procedure, results, and/or environment is recorded.}
{200pt}
%
\tpStepBULLSEYE{END RECORDING OF POST-TEST INFORMATION}
%from $TEXINPUTS; make a local copy and adjust as appropriate.
%
}%6%-STEPS
%%%%%%%%%%%%%%%%%%%%%%%%%%%%%%%%%%%%%%%%%


