%%% SVN stuff
\svnid{$Id: KNEAD_STP_Chapter_04_2_3.tex 153 2025-03-24 15:02:14Z Lewis $}

%%% SVN stuff
\svnid{$Id: STP_4.2.0_DIDINFO.tex 126 2024-07-14 03:51:40Z KneadProject $}

\DIDINFO{STP-4.2.0 :: This section shall be divided into the following subsections to describe the total scope of the planned testing.}

This section defines the plans for the \TestIdNameX validation efforts.

\subsubsection{Test Objective}
\label{loc:TestObjective\TestIdName}

The test objective for \TestIdNameX is to demonstrate one of three distinct areas:
\begin{enumerate}[itemindent=5pt,topsep=0pt,itemsep=0pt,partopsep=0pt, parsep=0pt]
	\item {\bf Installation Fit} demonstrates that the equipment can be installed into its designated space.
This test is often called the ``Form and Fit'' validation.
Such a validation ensures that the equipment meets the mechanical and electrical interfaces.
	\item {\bf Installed Operation} demonstrates that the equipment can be operated once installed into its designated space.
This test is often called the ``Function'' validation.
Once the equipment has been shown to capable of installation into its area, this test ensures that the equipment functions as expected.
The functionality may include basic operational checks as well as more esoteric assessments such as heat generation and cooling airflow.
	\item {\bf Component-Level Special Testing} demonstrates that the equipment meets special constraints it may endure once installed into its designated location.
While the ``Installed Operation'' test ensures that the equipment operates after installation, these tests ensure that the equipment can survive within the confines of the installation space.
For example, while general air flow may be assessed during the ``Installed Operation'' test, it may be known that the equipment may reach extreme temperatures or experience shock or vibrations when installed into the specified area.
These special tests validate that the equipment meets the specified operational and non-operational constraints expected within the installation space. 
\end{enumerate}


\subsubsection{Test Level}
\label{loc:TestLevels\TestIdName}

The test level for \TestIdNameX is \StageThree as defined by \citeStageTestingSTD.

\subsubsection{Test Type or Class}
\label{loc:TestType\TestIdName}

The test type or class for \TestIdNameX is development testing for installation fit or operation validation and environmental for special testing.

\subsubsection{Qualification Method}
\label{loc:TestQualificationMethod\TestIdName}

The test qualification method for \TestIdNameX generally is performed via test, but analysis may be performed as necessary to ensure each equipment item meets the design specifications.

\subsubsection{Traceability}
\label{loc:TestTraceability\TestIdName}

The test traceability for \TestIdNameX generally is included in the overall \RTVM.
For ``Form, Fit, or Function'' testing, the results are expected to map to related operational specifications.
For ``Special Tests'', the results are expected to map to the environmental specifications.
Both trace classes are expected to map System/Subsystem Specification (\SSS) implementation requirements to specific tests and/or test cases within a System Test Specification (\STS).

\subsubsection{Special Requirements}
\label{loc:TestSpecialRequirements\TestIdName}

The test special requirements for \TestIdNameX are defined as necessary in test procedures.
``Form, Fit, or Function'' testing may include references to overall assembly or operational set-up documentation.
``Special Testing'' details are expected to be performed at a specialized test facility that prepares its own test procedures based upon the facility's own quality management system and certifications.


\subsubsection{Data Recoding}
\label{loc:TestDataRecoding\TestIdName}

The test data recoding for \TestIdNameX is to record all inspection results in hard copy or electronic form.
Electronic form (e.g., a database application with bar code scanning capability) is preferred, but not required, to minimize labor effort and data entry errors.
For tests performed by a specialized test facility, they will follow their own processes for data recording and provide records to the development team as appropriate.

\subsubsection{Assumptions or Constraints}
\label{loc:TestAssumptionsOrConstraints\TestIdName}

The test assumptions or constraints for \TestIdNameX are defined in the associated inspection or test procedures.
As noted above, ``Special Testing'' details are expected to be performed at a specialized test facility.

\subsubsection{Safety, Security, and Privacy}
\label{loc:TestSafetySecurityPrivacy\TestIdName}

The test safety, security, and privacy for \TestIdNameX are defined in the associated inspection or test procedures.
Special care must be taken when test results indicate issues with security measured for the system.

\subsubsection{Tests and Test Cases}
\label{loc:TestsAndTestCases\TestIdName}

Tests and test cases for this test stage are:
\begin{enumerate}[itemindent=5pt,topsep=0pt,itemsep=0pt,partopsep=0pt, parsep=0pt]
	
	\item {\bf \TBD} Special tests will be determined during the design phase.
	%\begin{enumerate}[itemindent=5pt,topsep=0pt,itemsep=0pt,partopsep=0pt, parsep=0pt]
		%\item {\bf Radios} Material receipt inspections for all radio devices.
	%\end{enumerate}

\end{enumerate}
