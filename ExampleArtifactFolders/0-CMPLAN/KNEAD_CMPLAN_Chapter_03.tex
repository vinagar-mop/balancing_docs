%%% SVN stuff
\svnid{$Id: KNEAD_CMPLAN_Chapter_03.tex 126 2024-07-14 03:51:40Z KneadProject $}

\chapter{Planning}
\label{loc:Planning}
%%% SVN stuff
\svnid{$Id: CMPlan_3.0.0_DIDINFO.tex 126 2024-07-14 03:51:40Z KneadProject $}

\DIDINFO{CMPlan-3.0.0 :: This section provides an overview of the \CM planning.}



This section provides an overview of the configuration management process for \ThisSys.
This follows the outline provided at \url{https://acqnotes.com/acqnote/careerfields/configuration-management-plan}.


Scope and difficulty of the project: 

The project manager should know the project’s scope and complexity. This information helps decide how much configuration management is needed and which configuration management tasks should be part of the plan.

Requirements of Stakeholders: 

The project manager should find and involve important stakeholders, such as customers, users, and members of the project team. Understanding their needs and goals for configuration management helps make sure that the CMP fits their needs and helps the project succeed.

Configuration Items (CIs): 

CIs are the parts of a project that need to be controlled and handled throughout its lifecycle. The project manager should find and write down the CIs, including hardware, software, paperwork, etc. CIs must be correctly identified and put into groups for successful configuration management.

Identifying the configuration: 

The project manager should set up processes for naming and identifying CIs uniquely. This includes giving setups version numbers, revision codes, or other ways to track and distinguish them. Changes in the setup can be tracked and controlled when there are clear ways to identify them.

Change Management: 

The project manager should outline a change management method in the CMP. This process shows how changes to CIs will be asked for, looked at, accepted, put into place, and written down. It should have change control boards or groups that look at changes’ effects and risks and decide whether to approve them.

Documentation of the configuration: 

The CMP should discuss about’ the documentation needs for CIs. It should say what documents need to be made, in what format, and with how much information. The project manager should ensure all documentation is up-to-date, controlled by versions, and easy for all parties to access.

Configuration Audits: 

The CMP should have plans for configuration audits at different project steps. Audits ensure that the way CIs are set up matches the criteria and documentation. The project manager should plan and schedule these checks for quality and safety.

Configuration Management Tools: 

The project manager should analyze and choose the right configuration management tools to help implement the CMP. These tools help keep track of setup changes, manage versions, and make it easier for team members to work together.

Training and Communication: 

The project manager should think about teaching team members and stakeholders the CMP’s processes and rules for configuration management. The configuration management method is always followed when roles, responsibilities, and expectations are made clear.

Continuous Improvement: 

The CMP should be seen as a living record that can be viewed and changed over time. The project manager should ask the project team and partners for feedback so they can find places to improve and use what they’ve learned from other projects.
