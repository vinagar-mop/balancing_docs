%%% SVN stuff
\svnid{$Id: KNEAD_CMPLAN_Chapter_06.tex 126 2024-07-14 03:51:40Z KneadProject $}

\chapter{Change Management Process}
\label{loc:CM_Process}

This chapter describes the process to be followed in performing \CM for \ThisSystem.

\section{Baseline Process}
\label{loc:CM_ProcessBaseline}

This section lists the process to be followed to baseline all configuration items.

\section{Change Process}
\label{loc:CM_ProcessChange}

This section defines the formalized process for making a change to a baseline CM item. 
This process generally involves generation of a change request, an in-depth analysis of the impacts of the proposed change and then formal approval [or rejection] by the change management board. 
The plan defines how proposed changes are to be documented. 
\begin{itemize}[itemindent=5pt,topsep=0pt,itemsep=0pt,partopsep=0pt, parsep=0pt]
\item How they are submitted to the CM manager’s staff. 
\item How the staff prepares them for preliminary review by the change management board. 
\item How and when the board conducts this preliminary review. 
\item How the need [as determined by the board] for further analysis is recorded. 
\item How and when this analysis is presented to the board. 
\item How the disposition of the change request is documented and distributed by the staff.
\end{itemize}

\section{Process Accounting}
\label{loc:CM_ProcessAccounting}

This section describes the steps to be taken by the CM manager and staff that will keep the other participants in the project aware of the configuration of the various outputs and products of the project. 
They will follow these defined processes to make the current configuration of documents and products known, and available, in a timely manner. 
They will make the status of any proposed changes known as the changes are being considered by the change management board.
Today, for both documents and software products, this means having procedures for keeping and making available electronic files that contain the currently approved version of the item. 
They will make those files available to other project participants.

Configuration Status Accounting activities must be clearly addressed in the CM Plan. 
This includes the recording, reporting, and metrics pertaining to change states and status, change requests, change notices, and the impact of approved changes to releases of drawings, documentation, red-lining, analysis, reports, procedures, etc. associated with each CI as well as the date the change is incorporated in each affected effectivity.

\section{Process Auditing}
\label{loc:CM_ProcessAuditing}

This section defines the process, and the application of that process, for verifying the configuration of a hardware or software product. 
This process will be invoked during verification to ensure the product version being verified is known and is accurately described by its documentation. 
The processes describe how and by whom this audit is to be conducted.

A CM Plan must define Verification and Audit of the items chosen as CIs as well as their components. 
This includes verification and validation of the CIs against functional baseline requirements and sufficiency of functional requirements allocation to subsystems and components. 
In-process Functional Configuration Audits (FCA) as well as formal Physical Configuration Audit (PCA) criteria are defined against contractual requirements as are criteria for acceptance, delivery, maintenance and final disposition.


\section{Process Tools}
\label{loc:CM_ProcessTools}