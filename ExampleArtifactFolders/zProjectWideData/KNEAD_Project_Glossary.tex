%%% SVN stuff
\svnid{$Id: KNEAD_Project_Glossary.tex 126 2024-07-14 03:51:40Z KneadProject $}

%%% This is how the acronyms need to be formatted
%\CAC      & \begin{minipage}{\AcronymColumnWidth}{Common Access Card}\end{minipage}\\ \hline%

Communications		&	\KNEADglossaryEntryStart
{
Communication is information transfer, among users or processes, according to agreed conventions.
}\KNEADglossaryEntryClose

\Customer         & \KNEADglossaryEntryStart
{
The local government project lead who is acting as a general manager for the sponsor to ensure that the contractor team executes the project according to stakeholder goals.
}\KNEADglossaryEntryCloseLast

%%%IOC		&	\begin{minipage}{\GlossaryColumnWidth}{
%%%The first attainment of the minimum capability to effectively employ a weapon, item of equipment, or system of approved specific characteristics, and which is manned or operated by an adequately trained, equipped, and supported military unit or force.
%%%}\end{minipage}\\	\hline
%%%
%%%FOC		&	\begin{minipage}{\GlossaryColumnWidth}{
%%%The full attainment of the specific characteristics of the weapon, item of equipment, or system. FOC delivers the ability to use effectively a weapon, item of equipment, or system when operated by a trained, equipped, and supported military unit or force.
%%%}\end{minipage}\\	\hline
%%%
%%%Information Operations	&	\begin{minipage}{\GlossaryColumnWidth}{
%%%Actions taken to affect adversary information and information systems while defending one’s own information and information systems.
%%%}\end{minipage}\\	\hline
%%%
%%%Interoperability		&	\begin{minipage}{\GlossaryColumnWidth}{
%%%The condition achieved among communications-electronics systems or items of equipment when information or services can be exchanged directly and satisfactorily between them and their users.
%%%}\end{minipage}\\ \hline
%%%
%%%%MIL-STD-498      & \begin{minipage}{\GlossaryColumnWidth}{%
%%%%A process development standard that describes a well-defined software engineering process and associated artifacts (documents).
%%%%}\end{minipage}\\ \hline%
%%%
%%%Mode		&	\begin{minipage}{\GlossaryColumnWidth}{
%%%A set of technical parameters defining how information exchange occurs. These parameters may include frequency band, waveform, type encryption, data rates, duplex (half or full), channel bandwidth.
%%%}\end{minipage}\\ \hline
%%%
%%%Modulation		&	\begin{minipage}{\GlossaryColumnWidth}{
%%%The process by which some characteristic of a higher frequency wave is varied in accordance with the amplitude of a lower frequency wave.
%%%}\end{minipage}\\ \hline
%%%
%%%Multi-Channel		&	\begin{minipage}{\GlossaryColumnWidth}{
%%%A characteristic where an operator can simultaneously use more than one channel. Time-division multiplexing, frequency-division multiplexing, or phase-division multiplexing may accomplish multi-channel transmission.
%%%}\end{minipage}\\ \hline
%%%
%%%Multiband		&	\begin{minipage}{\GlossaryColumnWidth}{
%%%Multiband refers to operations in the frequency spectrum between limits of defined frequency bands for two or more channels, radios, or networks.
%%%}\end{minipage}\\ \hline
%%%
%%%\Operator      & \begin{minipage}{\GlossaryColumnWidth}{%
%%%The person who is running (operating) the system. 
%%%This term is preferred to ``user'' to help avoid confusion between the generic term of ``user'' which, in some cases, may represent the person upon which the system is being operated.
%%%}\end{minipage}\\ \hline%
%%%
%%%Set-Up Time		&	\begin{minipage}{\GlossaryColumnWidth}{
%%%The time required an operator to employ a communications device and have the ability to transmit or receive information. Normally the set-up time is evaluated from a cold start, although this may vary with the communications device and the task.
%%%}\end{minipage}\\ \hline
%%%
%%%\Sponsor      & \begin{minipage}{\GlossaryColumnWidth}{
%%%The government project supervisor that is responsible for overall project funding and success.
%%%}\end{minipage}\\ \hline%
%%%
%%%\Stakeholder      & \begin{minipage}{\GlossaryColumnWidth}{
%%%Any member of the project that has an interest in the project.
%%%}\end{minipage}\\ \hline%
%%%
%%%UHF		&	\begin{minipage}{\GlossaryColumnWidth}{
%%%The range of radio wave frequencies between 300 and 3000 \MHz. This band includes TV channels 14 to 36 as well as the GSM, 2G, 3G, and 4G LTE cellular bands.
%%%}\end{minipage}\\ \hline%
%%%
%%%VHF Low		&	\begin{minipage}{\GlossaryColumnWidth}{
%%%The range of radio wave frequencies between 49 and 108 \MHz. This band includes commercial FM radio broadcast as well as TV channels 2 to 6.
%%%}\end{minipage}\\ \hline%
%%%
%%%VHF High		&	\begin{minipage}{\GlossaryColumnWidth}{
%%%The range of radio wave frequencies between 169 and 216 \MHz. This band includes TV channels 7 to 13.
%%%}\end{minipage}\\ \hline%
%%%
%%%Waveform		&	\begin{minipage}{\GlossaryColumnWidth}{
%%%A waveform is the representation of a signal that includes the frequency, modulation type, message format, and/or transmission system.
%%%}\end{minipage}%\\ \hline%

%%%
%%% no \\\hline allowed on final line
%%%