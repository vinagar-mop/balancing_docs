A summary of the states is provided in Table~\ref{tab:States}.
See the formal specifications, if applicable, in the following sections for formal statement of the state requirements, and accompanying notes that provide further clarification on the meanings of the states.

\begin{table}[h]
	\begin{center}
		\begin{tabular}{|p{1.0in}|p{5.0in}|}
			\hline
			\hline
			\multicolumn{2}{|c|}{{\bf STATES}} \\
			\hline
				{\bf State Name} & {\bf Summary} \\
			\hline
			\hline
State 1 & Board bringup        \\ \hline
State 2 & Information screen   \\ \hline
State 3 & Help Screen          \\ \hline
State 4 & Game Play            \\
			\hline
			\hline
		\end{tabular}
		\caption{Summary of States for \ThisSystem}
		\label{tab:States}
	\end{center}
\end{table}

%%%
%%% include subsections and/or files with requirements for the states as needed
%%%

%%% \ONERQMTV[9] is macro for consistently formatted requirements
\ONERQMTV
% #1 is requirement Number
{\RqtNumberBase.1}
% #2 is Title
{Specifications 1.1: Board Bringup}
% #3 is requirement label (expected to be of form rqt:XXX)
{rqt:StateOne}
% #4 is Text of the specification
{
	When the system is turned on, all subsystem that are associated with the system must be initialized. Both the Sensor and Actuator should be communicating with the CPU.
}
% #5 is Status (S in {(T), (O), (I), (D)} listed by phase as \item [] S)
{
	\item [Phase 1] Threshold
}
% #6 is Acceptance
{
	The debugger must read out that the system as passed its calibration
}
% #7 is Traceability
{
	\item [1] This could be found in the YSM documentation.
}
% #8 is Notes; listed as enumeration \item ...
{
	\item The State-1 state generalizes the case where the system is in a good state.
}
% #9 is changebar version
{P1}
%%%%% end \ONERQMTV[9] macro


%%% \ONERQMTV[9] is macro for consistently formatted requirements
\ONERQMTV
% #1 is requirement Number
{\RqtNumberBase.2}
% #2 is Title
{Specification 1.2: Main Menu}
% #3 is requirement label (expected to be of form rqt:XXX)
{rqt:StateTwo}
% #4 is Text of the specification
{	
	Once all subsystems are initialized the LTDC screen shall display a main menu. The main menu shall display 2 tabs. The 2 tabs should be labeled as
		- A Help Screen
		- A Play Game 
}
% #5 is Status (S in {(T), (O), (I), (D)} listed by phase as \item [] S)
{
	\item [Phase 1] Threshold
}
% #6 is Acceptance
{
	Then Main Menu screen should be the first screen that pops up when the system is turned on.
}
% #7 is Traceability
{
	\item [1] This could be found in the YSM documentation.
}
% #8 is Notes; listed as enumeration \item ...
{
	\item The State-2 state generalizes the case where the system is in a good state.
}
% #9 is changebar version
{P1}
%%%%% end \ONERQMTV[9] macro


%%% \ONERQMTV[9] is macro for consistently formatted requirements
\ONERQMTV
% #1 is requirement Number
{\RqtNumberBase.3}
% #2 is Title
{Specification 1.3: Help Screen}
% #3 is requirement label (expected to be of form rqt:XXX)
{rqt:StateThree}
% #4 is Text of the specification
{
	Contains all of the information that is needed for playing the game.
}
% #5 is Status (S in {(T), (O), (I), (D)} listed by phase as \item [] S)
{
	\item [Phase 1] Threshold
}
% #6 is Acceptance
{
	The help screen should popup once the system option is selected from the main page is selected.
}
% #7 is Traceability
{
	\item [1] The screen must have a separate page to place information like that.
}
% #8 is Notes; listed as enumeration \item ...
{
	\item The State-3 state generalizes the case where the system is in a good state.
}
% #9 is changebar version
{P1}
%%%%% end \ONERQMTV[9] macro

%%% \ONERQMTV[9] is macro for consistently formatted requirements
\ONERQMTV
% #1 is requirement Number
{\RqtNumberBase.4}
% #2 is Title
{Specification 1.4: Game Play Screen}
% #3 is requirement label (expected to be of form rqt:XXX)
{rqt:StateThree}
% #4 is Text of the specification
{
	Once the button is pressed the system should reset the ball in the middle of the screen. From the menu page, the game screen should always start at level 0.
}
% #5 is Status (S in {(T), (O), (I), (D)} listed by phase as \item [] S)
{
	\item [Phase 1] Threshold
}
% #6 is Acceptance
{
	The game screen should popup once the systems option is selected from the main page.
}
% #7 is Traceability
{
	\item [1] The screen must have a separate page to place information like that.
}
% #8 is Notes; listed as enumeration \item ...
{
	\item The State-3 state generalizes the case where the system is in a good state.
}
% #9 is changebar version
{P1}
%%%%% end \ONERQMTV[9] macro