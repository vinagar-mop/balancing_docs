A summary of the sub-states is provided in Table~\ref{tab:SubStates}.
This table also provides a list of the states in which each sub-state is valid.
See the formal specifications, if applicable, in the following sections for formal statement of the sub-state requirements, and accompanying notes that provide further clarification on the meanings of the states.
\begin{table}[h]
	\begin{center}
		\begin{tabular}{|p{1.0in}|p{4.0in}|p{1.0in}|}
			\hline
			\hline
			\multicolumn{3}{|c|}{{\bf SUB-STATES}} \\
			\hline
				{\bf Sub-State Name} & {\bf Summary} & {\bf Valid States} \\ 
			\hline
			\hline
Sub State A & MEM init & State 1 \\ \hline
Sub State B & LCTD init & State 1 \\ \hline
Sub State C & Main Menu Screen & State 2 \\ \hline
Sub State D & Help Screen & State 3 \\ \hline
Sub State E & Starting Level Screen & State 4 \\ \hline
Sub State F & Randomizer & State 4 \\ \hline
Sub State G & Level Loader & State 4 \\ \hline
Sub State H & Reach End Goal & State 4 \\ 
			\hline
			\hline
		\end{tabular}
		\caption{Summary of Sub-States for \ThisSystem}
		\label{tab:SubStates}
	\end{center}
\end{table}

%%%
%%% include subsections and/or files with requirements for the sub-states as needed
%%%

%%% \ONERQMTV[9] is macro for consistently formatted requirements
\ONERQMTV
% #1 is requirement Number
{\RqtNumberBase.1}
% #2 is Title
{SubState A}
% #3 is requirement label (expected to be of form rqt:XXX)
{rqt:SubStateA}
% #4 is Text of the specification
{The Subsystems MEMS shall indicate they are fully configured.}
% #5 is Status (S in {(T), (O), (I), (D)} listed by phase as \item [] S)
{
	\item [Phase 1] Threshold
}
% #6 is Acceptance
{This requirement shall be verified by demonstration.}
% #7 is Traceability
{
	\item [1] The log file shall report that those systems have gone through there appropriate startup sequence.
}
% #8 is Notes; listed as enumeration \item ...
{
	\item The SubState-A substate generalizes the case where the system is \TBD.
}
% #9 is changebar version
{P1}
%%%%% end \ONERQMTV[9] macro

%%% \ONERQMTV[9] is macro for consistently formatted requirements
\ONERQMTV
% #1 is requirement Number
{\RqtNumberBase.1}
% #2 is Title
{SubState B}
% #3 is requirement label (expected to be of form rqt:XXX)
{rqt:SubStateB}
% #4 is Text of the specification
{The Subsystems LTDC shall indicate they are fully configured.}
% #5 is Status (S in {(T), (O), (I), (D)} listed by phase as \item [] S)
{
	\item [Phase 1] Threshold
}
% #6 is Acceptance
{This requirement shall be verified by demonstration.}
% #7 is Traceability
{
	\item [1] The log file shall report that those systems have gone through there appropriate startup sequence.
}
% #8 is Notes; listed as enumeration \item ...
{
	\item The SubState-A substate generalizes the case where the system is \TBD.
}
% #9 is changebar version
{P1}
%%%%% end \ONERQMTV[9] macro

%%% \ONERQMTV[9] is macro for consistently formatted requirements
\ONERQMTV
% #1 is requirement Number
{\RqtNumberBase.2}
% #2 is Title
{SubState C}
% #3 is requirement label (expected to be of form rqt:XXX)
{rqt:SubStateC}
% #4 is Text of the specification
{The game would have a start screen that would have options that the person would have.}
% #5 is Status (S in {(T), (O), (I), (D)} listed by phase as \item [] S)
{
	\item [Phase 1] Threshold
}
% #6 is Acceptance
{This requirement shall be verified by demonstration.}
% #7 is Traceability
{
	\item [N/A] This requirement is a base requirement.
}
% #8 is Notes; listed as enumeration \item ...
{
	\item The SubState-C substate generalizes the case where the system is \TBD.
}
% #9 is changebar version
{P1}
%%%%% end \ONERQMTV[9] macro


%%% \ONERQMTV[9] is macro for consistently formatted requirements
\ONERQMTV
% #1 is requirement Number
{\RqtNumberBase.3}
% #2 is Title
{SubState D}
% #3 is requirement label (expected to be of form rqt:XXX)
{rqt:SubStateD}
% #4 is Text of the specification
{The help screen contains information about what the system contains.}
% #5 is Status (S in {(T), (O), (I), (D)} listed by phase as \item [] S)
{
	\item [Phase 1] Threshold
}
% #6 is Acceptance
{This requirement shall be verified by demonstration.}
% #7 is Traceability
{
	\item [N/A] This requirement is a base requirement.
}
% #8 is Notes; listed as enumeration \item ...
{
	\item The SubState-D substate generalizes the case where the system is \TBD.
}
% #9 is changebar version
{P1}
%%%%% end \ONERQMTV[9] macro

%%% \ONERQMTV[9] is macro for consistently formatted requirements
\ONERQMTV
% #1 is requirement Number
{\RqtNumberBase.3}
% #2 is Title
{SubState E}
% #3 is requirement label (expected to be of form rqt:XXX)
{rqt:SubStateE}
% #4 is Text of the specification
{The Game starts with Level 1. All data such as position and orientation of the boards data is reset.}
% #5 is Status (S in {(T), (O), (I), (D)} listed by phase as \item [] S)
{
	\item [Phase 1] Threshold
}
% #6 is Acceptance
{This requirement shall be verified by demonstration.}
% #7 is Traceability
{
	\item [N/A] This requirement is a base requirement.
}
% #8 is Notes; listed as enumeration \item ...
{
	\item The SubState-E substate generalizes the case where the system is \TBD.
}
% #9 is changebar version
{P1}
%%%%% end \ONERQMTV[9] macro

%%% \ONERQMTV[9] is macro for consistently formatted requirements
\ONERQMTV
% #1 is requirement Number
{\RqtNumberBase.3}
% #2 is Title
{SubState F}
% #3 is requirement label (expected to be of form rqt:XXX)
{rqt:SubStateF}
% #4 is Text of the specification
{The Randomizer give a random location were user should place the ball as an end goal.}
% #5 is Status (S in {(T), (O), (I), (D)} listed by phase as \item [] S)
{
	\item [Phase 1] Threshold
}
% #6 is Acceptance
{This requirement shall be verified by demonstration.}
% #7 is Traceability
{
	\item [N/A] This requirement is a base requirement.
}
% #8 is Notes; listed as enumeration \item ...
{
	\item The SubState-F substate generalizes the case where the system is \TBD.
}
% #9 is changebar version
{P1}
%%%%% end \ONERQMTV[9] macro

%%% \ONERQMTV[9] is macro for consistently formatted requirements
\ONERQMTV
% #1 is requirement Number
{\RqtNumberBase.3}
% #2 is Title
{SubState G}
% #3 is requirement label (expected to be of form rqt:XXX)
{rqt:SubStateG}
% #4 is Text of the specification
{The ball in place in the center of the screen. }
% #5 is Status (S in {(T), (O), (I), (D)} listed by phase as \item [] S)
{
	\item [Phase 1] Threshold
}
% #6 is Acceptance
{This requirement shall be verified by demonstration.}
% #7 is Traceability
{
	\item [N/A] This requirement is a base requirement.
}
% #8 is Notes; listed as enumeration \item ...
{
	\item The SubState-G substate generalizes the case where the system is \TBD.
}
% #9 is changebar version
{P1}
%%%%% end \ONERQMTV[9] macro

%%% \ONERQMTV[9] is macro for consistently formatted requirements
\ONERQMTV
% #1 is requirement Number
{\RqtNumberBase.3}
% #2 is Title
{SubState H}
% #3 is requirement label (expected to be of form rqt:XXX)
{rqt:SubStateH}
% #4 is Text of the specification
{User win page shows on screen. The level is incremented and ball is restarted in the middle.}
% #5 is Status (S in {(T), (O), (I), (D)} listed by phase as \item [] S)
{
	\item [Phase 1] Threshold
}
% #6 is Acceptance
{This requirement shall be verified by demonstration.}
% #7 is Traceability
{
	\item [N/A] This requirement is a base requirement.
}
% #8 is Notes; listed as enumeration \item ...
{
	\item The SubState-H substate generalizes the case where the system is \TBD.
}
% #9 is changebar version
{P1}
%%%%% end \ONERQMTV[9] macro
