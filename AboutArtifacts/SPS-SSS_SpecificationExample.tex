%%% SVN stuff
\svnid{$Id: SPS-SSS_SpecificationFormatting_NoRefToHowToSections.tex 126 2024-07-14 03:51:40Z KneadProject $}

The system specifications are listed in a common table format as shown in Requirement~\ref{rqt:TableFormat}.
%%%This format allows for easy auto-generation from a database if the applicable fields are supplied.
%%%The regular structure of this format also allows for the {\em ParseRequirements.exe} tool to read all requirements and list them in a \CSV formatted file for analysis and review.

%%%%%%%%%%%%%%%%%%%%%%%%%%%%%%%%%%%%%%%%%
%% \ONERQMTV[9] (MULTIRQMTV[10]) is macro for consistently formatted requirements
\MULTIRQMTV
% #1 is requirement Number
{1.3.1}
% #2 is Title
{Specification Table Format}
% #3 is requirement label (expected to be of form rqt:XXX)
{rqt:TableFormat}
% #4 is Text of the specification
{The system requirements are listed in a common table format with the following parts:}
% #4a (#5 in multirqmt) is Multiple parts of the specification
{\small
 \item The first row of a table provides a unique number and a title for the requirement.

 \item The second row of the table provides the text of the requirement. This may extend to the next row for multi-part requirements. See the notes for more information on this practice. This line delineates \KPP or \KSA requirements, by adding ``\KPP'' or ``\KSA'' to the left side of this row.

 \item The next row of the table provides a list of the requirement specifics for multi-part requirements, or is omitted for single-part requirements.   
 
 \item The next row of the table provides the status for the items in the table. 

 \item The next row of the table provides the acceptance criteria. This row follows the form of ``This requirement shall be verified by V $\in$ \{inspection, demonstration, test, analysis\}''.

 \item The next row of the table provides the traceability of the requirement, which is a higher level document that calls out the need for a requirement. The structure of traceability is expected to be of the form ``This requirement is derived from MIL-STD-498~\cite{ref__MIL_STD_498} and ISO-12207~\cite{ref__ISO_12207}''. Note that the source is expected to be listed in the reference documents section.

 \item The final row of the table provides, if applicable, notes for the requirement that are not a formal part of the requirement but provide supporting information regarding the feature.
}
% #5 (#6 in multirqmt) is Status (S in {(T), (O), (I), (D)} listed by phase as \item [] S)
{
	\item [All] This format is active for all requirements in this document.
}
% #6  (#7 in multirqmt) is is Acceptance methodology
{This specification is not a testable requirement for the system; it is for demonstration purposes only.}
% #7 (#8 in multirqmt) is Traceability
{
\item [N/A] There is no traceability for this requirement.
}
% #8 (#9 in multirqmt) is Notes; listed as enumeration \item ...
{\small
	\item This table is generated using a \LaTeX command.
  \item The multi-part form is used here so that all of the rows are displayed.
  \item This formatting is not a testable requirement on the system, but rather, shows how the requirements are depicted in the document.
	\item The multi-part format allows for common items to be grouped together for conciseness and clarity. Each line of the multiple-parts is to be considered as a separate requirement, with a unique identifier (ID) that includes the point number added to the base requirements number. For example, requirement X.Y.Z with 3 parts would be unique IDs X.Y.Z.1, and so forth.
	\item The multi-part format is useful for tightly-coupled items. For example, a display monitor has specifications for diagonal size, resolution, brightness, contrast ratio, etc. By placing all such requirements into a single spot helps ensure that all of the closely related requirements are considered together.
}
% #9 (#10 in multirqmt) is changebar version
{P0}



