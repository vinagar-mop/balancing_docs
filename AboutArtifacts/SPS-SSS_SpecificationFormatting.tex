%%% SVN stuff
\svnid{$Id: SPS-SSS_SpecificationFormatting.tex 139 2025-06-13 20:26:50Z KneadProject $}

\subsection{Specification Formatting}
\label{loc:Intro_SpecFormatting}

%%% SVN stuff
\svnid{$Id: SPS-SSS_SpecificationFormatting_NoRefToHowToSections.tex 139 2025-06-13 20:26:50Z KneadProject $}

%%%subsection{Specification Formatting}
%%%\label{loc:Intro_SpecFormattingNoRef}

The specifications are listed and numbered by document sections. 
The fully qualified specification numbers include the sub-section in which it is contained. 
These specification numbers are tied to the document level thus they are numbered from 1 to N for each sub-section of the requirements section. 
This is done to allow for additions within a sub-sections without affecting the numbering in other sub-sections. 

Once a specification has been added, it cannot be deleted, only its status may be changed to ``inactive'' or ``deleted'' to preserve numbering.
The table format also allows for grouping of related specifications.
These grouped specification are also numbered sequentially and are also not removed if made inactive. 
Rather, they are marked in a ``strike-though'' font to denote that they are inactive.

This document allows for marking changes to specifications.
All specifications may be marked with a change bar.
This generally implies that one or more parts of a specification changed from the prior revision.
A note should be provided to indicate the reason for the change, and when, so that future versions of the document, which do not include the change bar, still have rationale included for the current value.

%%% SVN stuff
\svnid{$Id: SPS-SSS_SpecificationFormatting_NoRefToHowToSections.tex 126 2024-07-14 03:51:40Z KneadProject $}

The system specifications are listed in a common table format as shown in Requirement~\ref{rqt:TableFormat}.
%%%This format allows for easy auto-generation from a database if the applicable fields are supplied.
%%%The regular structure of this format also allows for the {\em ParseRequirements.exe} tool to read all requirements and list them in a \CSV formatted file for analysis and review.

%%%%%%%%%%%%%%%%%%%%%%%%%%%%%%%%%%%%%%%%%
%% \ONERQMTV[9] (MULTIRQMTV[10]) is macro for consistently formatted requirements
\MULTIRQMTV
% #1 is requirement Number
{1.3.1}
% #2 is Title
{Specification Table Format}
% #3 is requirement label (expected to be of form rqt:XXX)
{rqt:TableFormat}
% #4 is Text of the specification
{The system requirements are listed in a common table format with the following parts:}
% #4a (#5 in multirqmt) is Multiple parts of the specification
{\small
 \item The first row of a table provides a unique number and a title for the requirement.

 \item The second row of the table provides the text of the requirement. This may extend to the next row for multi-part requirements. See the notes for more information on this practice. This line delineates \KPP or \KSA requirements, by adding ``\KPP'' or ``\KSA'' to the left side of this row.

 \item The next row of the table provides a list of the requirement specifics for multi-part requirements, or is omitted for single-part requirements.   
 
 \item The next row of the table provides the status for the items in the table. 

 \item The next row of the table provides the acceptance criteria. This row follows the form of ``This requirement shall be verified by V $\in$ \{inspection, demonstration, test, analysis\}''.

 \item The next row of the table provides the traceability of the requirement, which is a higher level document that calls out the need for a requirement. The structure of traceability is expected to be of the form ``This requirement is derived from MIL-STD-498~\cite{ref__MIL_STD_498} and ISO-12207~\cite{ref__ISO_12207}''. Note that the source is expected to be listed in the reference documents section.

 \item The final row of the table provides, if applicable, notes for the requirement that are not a formal part of the requirement but provide supporting information regarding the feature.
}
% #5 (#6 in multirqmt) is Status (S in {(T), (O), (I), (D)} listed by phase as \item [] S)
{
	\item [All] This format is active for all requirements in this document.
}
% #6  (#7 in multirqmt) is is Acceptance methodology
{This specification is not a testable requirement for the system; it is for demonstration purposes only.}
% #7 (#8 in multirqmt) is Traceability
{
\item [N/A] There is no traceability for this requirement.
}
% #8 (#9 in multirqmt) is Notes; listed as enumeration \item ...
{\small
	\item This table is generated using a \LaTeX command.
  \item The multi-part form is used here so that all of the rows are displayed.
  \item This formatting is not a testable requirement on the system, but rather, shows how the requirements are depicted in the document.
	\item The multi-part format allows for common items to be grouped together for conciseness and clarity. Each line of the multiple-parts is to be considered as a separate requirement, with a unique identifier (ID) that includes the point number added to the base requirements number. For example, requirement X.Y.Z with 3 parts would be unique IDs X.Y.Z.1, and so forth.
	\item The multi-part format is useful for tightly-coupled items. For example, a display monitor has specifications for diagonal size, resolution, brightness, contrast ratio, etc. By placing all such requirements into a single spot helps ensure that all of the closely related requirements are considered together.
}
% #9 (#10 in multirqmt) is changebar version
{P0}





The status designations for each specification Status $\in$ \{(T)hreshold, (O)bjective, (I)nactive, (D)eleted\} are based on the following criteria.
\begin{my_description}
{
\item [\OneRqmtThreshold] Items marked ``\OneRqmtThreshold'' are driven by the project threshold needs that must be met in the specified phase.

\item [\OneRqmtObjective] Items marked ``\OneRqmtObjective'' are objective goals of the system in the specified phase. These requirements may stay (O) for all listed phases or may transition from (O) to (T) in future phases. This provides hints as to future expansion of system capabilities so the design can account for the feature without significant later rework.

\item [\OneRqmtInactive] Items marked ``\OneRqmtInactive'' are requirements that are not currently to be met by the system in the specified phase. Unlike `\OneRqmtObjective'' requirements, (I) requirements may be in limbo in terms of certain details but their inclusion may also provide hints as to future expansion of system capabilities. 

\item [\OneRqmtDeleted] Requirements that are not to be met by the system are marked by ``\OneRqmtDeleted''. Use of this status, vice removal of the requirement text, preserves the numbering of subsequent requirements and notes that the requirement once was invoked.
The rationale for the deletion should be included in the notes section. 
}
\end{my_description}

The status may include the applicable versions in which the required feature is supported.
For example, when a phased development approach is used, a format of [Phase-X] Status can be employed, with multiple phases listed, to indicate when the requirement is to be included in the system.

\if@showreqnotes
Another major advantage of the table format is the ``Notes" section.
As specifications are developed, there will be many issues to be resolved.
And, once issues are clarified, tracking the rationale for the decision is just as important as recording the answer~\cite{ref__Brooks_MMM}.
Thus the notes section helps the reader and the writer.
The writer has a logically grouped place to put notes for a specification set and the reader can easily find them without having to refer to footnotes, separated sections, or external documentation.
A side benefit of the \LaTeX{} formatting is that the notes can be easily omitted in document generation for presentation of the specifications to customers and other external readers, thus protecting possibly sensitive information.
\fi% end show notes


External tools have been written that allow for automatic generation of other documentation.
Specially, data for chapters and appendices that follow the requirement specifications, can be gleaned automatically to ensure integrity between the sections of the documents.
In addition, the listing of \KPP and \KSA values into a ``B-spec'' can be automated.
Finally, the full set of requirements, and the associated attributes, are exported to a comprehensive \CSV file for import into \MBSE tools such as \DOORS or \CAMEO.


This table approach offers other advantages besides automated parsing for import into tools.
As can be seen in Table~\ref{rqt:TableFormat}, and in all the specifications, this format groups all information for each specification into a separable and easily viewed structure.
The document sections and subsections provide a logical grouping of the specifications but the table allows all pertinent information to be grouped, vice being split across major sections of the document.
This grouping allows for easier presentation since each grouping is similar to a ``PowerPoint'' presentation slide.
And, as will be seen in Section~\ref{loc:Intro_HowToReadSpecifications}, it can help the writer organize specifications.
The approach also allows for a ``List of Specifications'' to be generated.
Each table is listed in the list of specifications so that each high level grouping can be quickly located from the list.
Of course, the tables are located in the appropriate sections as noted in Section~\ref{loc:DocOverview_ArtifactFormat} so they can be found in that manner as well.


Another major advantage of the table format is the ``Notes'' section.
As specifications are developed, there will be many issues to be resolved.
And, once issues are clarified, tracking the rationale for the decision is just as important as recording the answer~\cite{ref__Brooks_MMM}.
Thus the notes section helps the reader and the writer.
The writer has a logically grouped place to put notes for each specification and the reader can easily find them without having to refer to footnotes, separated sections, or external documentation.

